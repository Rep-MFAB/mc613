\section{Introdução}

Um dos métodos mais básicos de implementar um algoritmo de 
aprendizagem de máquina é a regressão linear, que consiste em minimizar
pelo método gradiente e na função de custo usando o método dos 
mínimos quadrados.

A aprendizagem de máquina consiste em 2 etapas, a etapa de treinamento
e a de aplicação. Na etapa de treinamento temos uma lista de entradas e
saídas esperadas, e otimizamos uma função para que ela tenha o menor
erro possível. Na etapa de aplicação usamos a função otimizada em 
novas entradas para produzir uma saída ótima.

A regressão linear parte da função hipótese:

$$h_\theta (x) = \sum_{i=0}^{n} \theta_i x_i$$

Onde $x_i$ são as entradas, $n$ é o número de parâmetros e 
$\theta_i$ são os argumentos que queremos encontrar

A função de custo, que deve ser minimizada, é baseada no método
dos mínimos quadrados, sendo a função:

$$ J(\theta) = \frac{1}{2n}\sum_{i=i}^{n}(h_\theta (x_i) - y_i)^2$$

Onde $y_i$ é o resultado esperado para $x_i$

O método gradiente consiste em iterar sobre:

$$ \theta_i := \theta_i - 
    \alpha \frac{\partial}{\partial \theta_i} J(\theta)$$

Onde $\alpha$ é o índice de aprendizado (quanto maior mais 
rápido o aprendizado, quanto menor mais preciso).

O sistema funcionará com um teclado e um monitor.

As teclas terão as seguintes funcionalidades:

\begin{itemize}
    \item Números - valores de x ou y
    \item Espaço - ir para o próximo x
    \item Tab - alternar entre x e y
    \item Enter - adicionar valores em treino
    \item Ctrl+Enter - testar
\end{itemize}

O monitor irá mostrar a resposta e o custo atual.